\documentclass[../report]{subfiles}
\setcounter{section}{0}
\begin{document}

本章では、本プロジェクトの背景と目的を説明する。

\section{認知症の現状}
\begin{hissu}
プロジェクトの分野の状況や、類似プロジェクトがあればその状況を
記述する。前年度からの継続課題ならば、前年度の内容も記述する。
\end{hissu}
一般にカレーという料理は家庭でよく作られる。これまで多くの人が
おいしいカレーの作り方について試行錯誤してきている。函館の特産
品を用いた一般料理が少ない。前年度は、省略。
\bunseki{北海太郎}

\section{現状における問題点}
\begin{hissu}
現状のままでは存在する問題点について、記述する。
いわば当プロジェクトの存在意義
\end{hissu}
作るたびにカレーの味が変わる。いつもおいしいものができるとは限らない。
\bunseki{未来花子}

\section{目的}
\begin{hissu}
 上述の問題点を解決すべく当プロジェクトの掲げる課題の概要を述べる。
\end{hissu}
地域の特色を生かしたおいしいカレーの作り方が課題。
\bunseki{未来太郎}

\end{document}
