\documentclass[../report]{subfiles}
\setcounter{section}{0}
\begin{document}

本章では,学びと今後の展望について説明する.

\section{学び}
 本グループがこれまでのシステム開発の中で学んだこととして三つ挙げられる.
 一つ目は,これまでに考えたアイデアにとらわれずに,新しいアイデアを考えることの大切さである.ロボット開発ワークショップを終えるまで,色々なレビューや学びの機会があったにも関わらずアイデアに幅が出なかった.その解決策として,考え方を変え,これまでの学びや頂いたご意見を参考に,新たにアイデア出しをした.その結果,システムの方向性を決めることができた.具体的に参考にしたご意見を以下に示す.

\begin{itemize}
    \item[] 京都府立医科大学成本医師から
    \item 映像だったり普段の何気ない選択などが可視化されて医療従事者に提供されると良いと思う.
    \item 普段の傾向をつかむことやプロセスの透明化というのは医療選択する手段であるが,これをICTの力でアシストしてほしい.
\end{itemize}


 二つ目は,ICTに不慣れな高齢者への配慮の重要さである.本グループはシステム案を考える過程において,システムの使いやすさや高齢者への負担についての指摘を多数頂いた.このことから,本グループはシステム案を見直し,操作を簡単•自動化させることで,使いやすいシステムになるように心掛けた.
 三つ目は,認知症予防を実現する具体的な方法である.本グループは検出したデータをいかに加工して価値を付与できるかにこだわりすぎていた.しかし京都府立医科大学の成本医師によると,生活パターンのデータを高齢者自身に見てもらうだけでも,認知症予防に効果があるということである.本グループはこの学びを活かし,新たなシステム案を考えた.
\bunseki{頼亜弥}


\section{今後の展望}
本グループの今後の展望として,主に三つ挙げられる.

 一つ目は,「認知症予防のための食習慣改善システム」の内容をより掘り下げることである.本グループは中間発表と今までに頂いたシステム案のレビューや意見を受け止め,システムを改善していきたいと考えている.

 二つ目は,このシステムを実際に高齢者とその家族に使ってもらい,フィードバックを受けることである.「認知症予防のための食習慣改善システム」について,実際に検定した結果やデータはまだない状態である.このシステムの有効性を検証するため,高齢者やその家族に使ってもらう必要性があると考えた.

 三つ目は,開発のスケジュール管理をしっかり行うことである.本グループは中間発表までにシステム案を考案し,考案したシステム案を改善していた.これからのシステム開発を効率的に進めるために,今後他の生活習慣に対するアプローチをどうしていくかということも含め,スケジュール管理をしっかり行っていきたいと考えている.
\bunseki{頼亜弥}

\end{document}
