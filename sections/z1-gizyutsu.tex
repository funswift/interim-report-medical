\documentclass[../report]{subfiles}
\begin{document}

\begin{description}
    \item[LaTeX] \mbox{} \\
        グループ報告書作成にあたり、TeXを利用した。
        TeXはマークアップ言語で記述された文章構造から組版を行うソフトウェアである。
        数学者であるDonald E.Knuthが開発した。
        TeXの優れている点は、テキストファイルで編集を行うため、後からフォントや節の番号を編集する作業が容易に行えるという点である。
    \item[Git] \mbox{} \\
        グループ報告書作成にあたり、Gitを利用した。
        Gitはプログラムのソースコードなどの変更履歴を記録・追跡するための分散型バージョン管理システムである。
        Linuxの開発者であるリーナス・トーバルズが開発した。
        Gitの優れた点は、SVNなどの中央集権型のバージョン管理システムとは異なり、各ユーザーが全履歴を保持する分散型である。
        従ってサーバーにホスティングしていたリポジトリが消失しても、手元のリポジトリを元に回復させることが出来る。
\end{description}

\end{document}
